\documentclass[11pt]{article}

%%%%%%%%%%%%%%%%%%%%%%%%%%%%%%%%%%%%%%%%%%%%%%%%%%%%%%%%%%%%%%%%%%%%%%%%%%%%%%%%%%%%%%%%%%%%%%%%
%%%%%%%%%%%%%%%%%%%%%%%%%%%%%%%%%%%%%%%%%%%%%%%%%%%%%%%%%%%%%%%%%%%%%%%%%%%%%%%%%%%%%%%%%%%%%%%%

% MATH 167 CRN 30790
% WINTER QUARTER 2022


% Created by Xiaotie Chen, April 9th 2021

%%%%%%%%%%%%%%%%%%%%%%%%%%%%%%%%%%%%%%%%%%%%%%%%%%%%%%%%%%%%%%%%%%%%%%%%%%%%%%%%%%%%%%%%%%%%%%%%
%%%%%%%%% Load General Packages %%%%%%%%%%%%%%%%%%%%%%%%%%%%%%%%%%%%%%%%%%%%%%%%%%%%%%%%%%%%%%%%
%%%%%%%%%%%%%%%%%%%%%%%%%%%%%%%%%%%%%%%%%%%%%%%%%%%%%%%%%%%%%%%%%%%%%%%%%%%%%%%%%%%%%%%%%%%%%%%%
%%%%%%%%%%%%%%%%%%%%%%%%%%%%%%%%%%%%%%%%%%%%%%%%%%%%%%%%%%%%%%%%%%%%%%%%%%%%%%%%%%%%%%%%%%%%%%%%

\usepackage{amsmath,amssymb,amsthm,fullpage}
\usepackage{listings}
\usepackage{xcolor}
\usepackage{graphicx}
\usepackage{subfigure}

\usepackage{float}

%%%%%%%%%%%%%%%%%%%%%%%%%%%%%%%%%%%%%%%%%%%%%%%%%%%%%%%%%%%%%%%%%%%%%%%%%%%%%%%%%%%%%%%%%%%%%%%%
%%%%%%%%% Set Up Page Parameters %%%%%%%%%%%%%%%%%%%%%%%%%%%%%%%%%%%%%%%%%%%%%%%%%%%%%%%%%%%%%%%
%%%%%%%%%%%%%%%%%%%%%%%%%%%%%%%%%%%%%%%%%%%%%%%%%%%%%%%%%%%%%%%%%%%%%%%%%%%%%%%%%%%%%%%%%%%%%%%%

% Please refer to Prof. Puckett 's template for explanations

\setlength{\textheight}{ 620 pt}
\setlength{\voffset}{-20 pt}
\setlength{\headsep}{12 pt}
\setlength{\headheight}{14 pt}
\setlength{\topskip}{12 pt}
\setlength{\footskip}{30 pt}

%%%%%%%%%%%%%%%%%%%%%%%%%%%%%%%%%%%%%%%%%%%%%%%%%%%%%%%%%%%%%%%%%%%%%%%%%%%%%%%%%%%%%%%%%%%%%%%%
%%%%%%%% SET HEADERS AND FOOTERS %%%%%%%%%%%%%%%%%%%%%%%%%%%%%%%%%%%%%%%%%%%%%%%%%%%%%%%%%%%%%%%
%%%%%%%%%%%%%%%%%%%%%%%%%%%%%%%%%%%%%%%%%%%%%%%%%%%%%%%%%%%%%%%%%%%%%%%%%%%%%%%%%%%%%%%%%%%%%%%%
% "L" stands for "Left", "C" for "Centered", "R" for "Right", "O" for Odd and "E" for Even page

\usepackage{fancyhdr}
\pagestyle{fancy}
\fancyhead{}

\fancyhead[L]{\textbf{NAME: Ishita Dutta}} % TYPE YOUR NAME HERE!
\fancyhead[C]{\textbf{HW 01}}              % TYPE THE ASSGNMENT NAME HERE!
\fancyhead[R]{\textbf{SID: 918193342}}     % TYPE YOUR STUDENT ID HERE!

\fancyfoot[L]{CRN 38665}
\fancyfoot[C]{--~\thepage~--}
\fancyfoot[R]{\today}

%%%%%%%%%%%%%%%%%%%%%%%%%%%%%%%%%%%%%%%%%%%%%%%%%%%%%%%%%%%%%%%%%%%%%%%%%%%%%%%%%%%%%%%%%%%%%%%%
%%%%%%%% START DOCUMENT
%%%%%%%%%%%%%%%%%%%%%%%%%%%%%%%%%%%%%%%%%%%%%%%%%%%%%%%%%%%%%%%%%%%%%%%%%%%%%%%%%%%%%%%%%%%%%%%%

\begin{document}

 % Create a numbered list

 \begin{enumerate}

%%%%%%%%%%%%%%%%%%%%%%%%%%%%%%%%%%%%%%%%%%%%%%%%%%%%%%%%%%%%%%%%%%%%%%%%%%%%%%%%%%%%%%%%%%%%%%%%
 %%%%%%%% Problem 1
%%%%%%%%%%%%%%%%%%%%%%%%%%%%%%%%%%%%%%%%%%%%%%%%%%%%%%%%%%%%%%%%%%%%%%%%%%%%%%%%%%%%%%%%%%%%%%%%

    % Always indent your code whether it's LaTeX or Matlab. It make is easier and clearer to
    % read, and this leads to fewer mistakes. I always use 2 spaces. I think Overleaf uses
    % 4 spaces for LaTeX code.

    \item \label{Problem_01} (50 pts)

      % Create a list at the inside Problem 01. By default it will be labeled with lowercase
      % letters (a), (b) ...

      \begin{enumerate}

        \item If you want to write some math equations in \LaTeX, you can use
          $$ A=QR. $$ or
          \begin{equation*}
            A = QR.
          \end{equation*}

          Note that these equations are `broken out' in the sense that they do not appear
          in the sentence but separately in their own place on the page.

          \vskip 06pt

          Also note that your can use two dollar signs at the beginning and again at the
          end of a `broken out' equation like this
          \begin{verbatim}
            $$ A = QR. $$
          \end{verbatim}
           as shorthand for
          \begin{verbatim}
            \begin{equation*}
              A = QR.
            \end{equation*}
          \end{verbatim}
          Both equations typeset the same way.

          \vskip 06pt

          If you only want to write some simple expressions in line, you can use $A=QR$ .

        \item You should always use boldface for vectors to distinguish them from scalars.
          For example,
          $$
            \mathbf{v}= [v_1, v_2, v_3],
          $$
          here $\mathbf{v}$ is a vector in $\mathbf{R}^3$, and $v_1, v_2, v_3$ are scalars.

         \vskip 06pt
        Use capital letters for matrices and boldface lower-case letters for vectors.

        Have you noticed that we are staring a new paragraph using
        $\backslash\text{vskip~06pt}$?
        However, you do not have to use  $\backslash\text{vskip~06pt}$ everywhere, just when
        you want to force \LaTeX~to make a paragraph with a 06 points of space between the
        paragraphs. (72 points or 72pt = 01 inch)

        \vskip 06pt

        \item You can code the pseudoinverse of a matrix A with: $A^\dagger$. To represent
        the transpose simply use $A^T$.  To write $x$ to the power $n$ use $x^n$.  You can
        replace $n$ with any number, but if there is more than one digit use curly braces;
        e.g., like this $x^{-1}$ or this $x^{23}$.  You can replace the scalar $x$ with a
        vector too (e.g. $\mathbf{v}^T$) or any other mathematical symbol you want.

        \vskip 06pt

        \item Here is how you can represent a matrix or a column vector in \LaTeX:
           \begin{equation*}
               A = \begin{bmatrix}
                         1 & 2
                         0 & 0
                   \end{bmatrix}  ,
               \mathbf{b} = \begin{bmatrix}
                         1
                         0
                   \end{bmatrix}
           \end{equation*}

        \vskip 06pt
        \item Note: if you want to type braces, use \{ \}

        \vskip 06pt
             \item Type your answer for Part f) of Problem 1 here

        \end{enumerate}



%%%%%%%%%%%%%%%%%%%%%%%%%%%%%%%%%%%%%%%%%%%%%%%%%%%%%%%%%%%%%%%%
 %%%%%%%% Problem 2
%%%%%%%%%%%%%%%%%%%%%%%%%%%%%%%%%%%%%%%%%%%%%%%%%%%%%%%%%%%%%%%%%%%%%%%%%%%%%%%%%%%%%%%%%%%%%%%%

    \item \label{Problem_02} (15 pts)

        \begin{enumerate}

             \item Type your answer for Part a) of Problem 2 here

             \vskip 06pt
             \item Type your answer for Part b) of Problem 2 here

             \vskip 06pt
             \item Type your answer for Part c) of Problem 2 here

        \end{enumerate}

   %%%%%%%%%%%%%%%%%%%%%%%%%%%%%%%%%%%%%%%%%%%%%%%%%%%%%%%%%%%%%%%%%%%%%%%%%%%%%%%%%%%%%%%%%%%%%%%%
 %%%%%%%% Problem 3
%%%%%%%%%%%%%%%%%%%%%%%%%%%%%%%%%%%%%%%%%%%%%%%%%%%%%%%%%%%%%%%%%%%%%%%%%%%%%%%%%%%%%%%%%%%%%%%%

    \item \label{Problem_03} (65 pts)

        \begin{enumerate}


             \item Reminder: always use BOLDFACE for vectors! For example, given
             $$
                A= [\mathbf{a}_1, \mathbf{a}_2, \mathbf{a}_3, \mathbf{a}_4],
             $$
             you should use boldface for $\mathbf{a}_1, \mathbf{a}_2, \mathbf{a}_3, \mathbf{a}_4$ as they are column VECTORS of matrix $A$.

             \vskip 06pt
             \item Type your answer for Part b) of Problem 3 here

             \vskip 06pt
             \item Type your answer for Part c) of Problem 3 here

             \vskip 06pt
             \item Type your answer for Part d) of Problem 3 here

             \vskip 06pt
             \item Please Upload your MATLAB code for part e) to CANVAS. Check your Assignment files on detailed instructions. You do not need to type anything for Part e) of Problem 3

        \end{enumerate}


        %%%%%%%%%%%%%%%%%%%%%%%%%%%%%%%%%%%%%%%%%%%%%%%%%%%%%%%%%%%%%%%%%%%%%%%%%%%%%%%%%%%%%%%%%%%%%%%%
 %%%%%%%% Problem 4
%%%%%%%%%%%%%%%%%%%%%%%%%%%%%%%%%%%%%%%%%%%%%%%%%%%%%%%%%%%%%%%%%%%%%%%%%%%%%%%%%%%%%%%%%%%%%%%%

    \item \label{Problem_04} (50 pts)

      \begin{enumerate}

      \item Type your answer for Part a) of Problem 4 here

        \vskip 06pt
        \item Type your answer for Part b) of Problem 4 here

        \vskip 06pt
        \item Type your answer for Part c) of Problem 4 here

        \vskip 06pt
        \item Type your answer for Part d) of Problem 4 here

        \vskip 06pt
        \item Type your answer for Part e) of Problem 4 here

        \vskip 06pt
        \item Type your answer for Part f) of Problem 4 here

        \vskip 06pt
        \item Type your answer for Part g) of Problem 4 here

        \vskip 06pt
        \item Type your answer for Part h) of Problem 4 here

        \vskip 06pt
        \item Type your answer for Part i) of Problem 4 here

        \vskip 06pt
        \item Type your answer for Part j) of Problem 4 here

      \end{enumerate}

%%%%%%%%%%%%%%%%%%%%%%%%%%%%%%%%%%%%%%%%%%%%%%%%%%%%%%%%%%%%%%%%
 %%%%%%%% Problem 5
%%%%%%%%%%%%%%%%%%%%%%%%%%%%%%%%%%%%%%%%%%%%%%%%%%%%%%%%%%%%%%%%%%%%%%%%%%%%%%%%%%%%%%%%%%%%%%%%

    \item \label{Problem_05} (30 pts)

        Type your answer for Problem 5 here

%%%%%%%%%%%%%%%%%%%%%%%%%%%%%%%%%%%%%%%%%%%%%%%%%%%%%%%%%%%%%%%%
 %%%%%%%% Problem 6
%%%%%%%%%%%%%%%%%%%%%%%%%%%%%%%%%%%%%%%%%%%%%%%%%%%%%%%%%%%%%%%%%%%%%%%%%%%%%%%%%%%%%%%%%%%%%%%%

    \item \label{Problem_06} (30 pts)

        \begin{enumerate}

             \item Type your answer for Part a) of Problem 6 here

             \vskip 06pt
             \item Type your answer for Part b) of Problem 6 here


        \end{enumerate}

%%%%%%%%%%%%%%%%%%%%%%%%%%%%%%%%%%%%%%%%%%%%%%%%%%%%%%%%%%%%%%%%%%%%%%%%%%%%%%%%%%%%%%%%%%%%%%%%
 %%%%%%%% Problem 7
%%%%%%%%%%%%%%%%%%%%%%%%%%%%%%%%%%%%%%%%%%%%%%%%%%%%%%%%%%%%%%%%%%%%%%%%%%%%%%%%%%%%%%%%%%%%%%%%

    \item \label{Problem_07} (30 pts)

        \begin{enumerate}

             \item Type your answer for Part a) of Problem 7 here

             \vskip 06pt
             \item Type your answer for Part b) of Problem 7 here

             \vskip 06pt
             \item Type your answer for Part c) of Problem 7 here

             \vskip 06pt
             \item Type your answer for Part d) of Problem 7 here

        \end{enumerate}

%%%%%%%%%%%%%%%%%%%%%%%%%%%%%%%%%%%%%%%%%%%%%%%%%%%%%%%%%%%%%%%%
 %%%%%%%% Problem 8
%%%%%%%%%%%%%%%%%%%%%%%%%%%%%%%%%%%%%%%%%%%%%%%%%%%%%%%%%%%%%%%%%%%%%%%%%%%%%%%%%%%%%%%%%%%%%%%%

    \item \label{Problem_08} (25 pts)

        \begin{enumerate}

             \item Type your answer for Part a) of Problem 8 here

             \vskip 06pt
             \item Type your answer for Part b) of Problem 8 here

             \vskip 06pt
             \item Type your answer for Part c) of Problem 8 here

             \vskip 06pt
             \item Type your answer for Part d) of Problem 8 here

        \end{enumerate}


\end{enumerate}

\end{document}