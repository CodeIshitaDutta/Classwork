\documentclass[twoside, letter]{article}
\setlength{\oddsidemargin}{0.01 in}
\setlength{\evensidemargin}{0.01 in}
\setlength{\topmargin}{-0.6 in}
\setlength{\textwidth}{6.5 in}
\setlength{\textheight}{8.5 in}
\setlength{\headsep}{0.75 in}
\setlength{\parindent}{0 in}
\setlength{\parskip}{0.1 in}

\usepackage{natbib}
\usepackage{hyperref}
\usepackage{listings}
\usepackage{xcolor}
\definecolor{codegreen}{rgb}{0,0.6,0}
\definecolor{codegray}{rgb}{0.5,0.5,0.5}
\definecolor{codepurple}{rgb}{0.58,0,0.82}
\definecolor{backcolour}{rgb}{0.95,0.95,0.92}
\usepackage{pdfpages}

\lstdefinestyle{mystyle}{
    backgroundcolor=\color{backcolour},   
    commentstyle=\color{codegreen},
    keywordstyle=\color{magenta},
    numberstyle=\tiny\color{codegray},
    stringstyle=\color{codepurple},
    basicstyle=\ttfamily\footnotesize,
    breakatwhitespace=false,         
    breaklines=true,                 
    captionpos=b,                    
    keepspaces=true,                 
    numbers=left,                    
    numbersep=5pt,                  
    showspaces=false,                
    showstringspaces=false,
    showtabs=false,                  
    tabsize=2
}

\usepackage{caption}
\lstset{style=mystyle}
%
% ADD PACKAGES here:
%



\usepackage{amsmath,amsfonts,amssymb,graphicx,mathtools,flexisym}

%
% The following commands set up the lecnum (lecture number)
% counter and make various numbering schemes work relative
% to the lecture number.
%
\newcounter{lecnum}
\renewcommand{\thepage}{\thelecnum-\arabic{page}}
\renewcommand{\thesection}{\thelecnum.\arabic{section}}
\renewcommand{\theequation}{\thelecnum.\arabic{equation}}
\renewcommand{\thefigure}{\thelecnum.\arabic{figure}}
\renewcommand{\thetable}{\thelecnum.\arabic{table}}

%
% The following macro is used to generate the header.
%
\newcommand{\lecture}[4]{
   \pagestyle{myheadings}
   \thispagestyle{plain}
   \newpage
   \setcounter{lecnum}{#1}
   \setcounter{page}{1}
   \noindent
   \begin{center}
   \framebox{
      \vbox{\vspace{2mm}
    \hbox to 6.28in { {\bf STA 141C - Big Data \& High Performance Statistical Computing
	\hfill Winter 2023} }
       \vspace{4mm}
       \hbox to 6.28in { {\Large \hfill Homework 3 \hfill} }
       \vspace{2mm}
       %\hbox to 6.28in { {\it Lecturer: #3 \hfill Scribes: #4} }
       \hbox to 6.28in { {\it Lecturer: #2 \hfill  Due March 03, 2023} }
      \vspace{2mm}}
   }
   \end{center}
   \markboth{Homework #1: #2}{Homework #1: #2}

  % {\bf Note}: {\it LaTeX template courtesy of UC Berkeley EECS dept.}

}
%
% Convention for citations is authors' initials followed by the year.
% For example, to cite a paper by Leighton and Maggs you would type
% \cite{LM89}, and to cite a paper by Strassen you would type \cite{S69}.
% (To avoid bibliography problems, for now we redefine the \cite command.)
% Also commands that create a suitable format for the reference list.
\renewcommand{\cite}[1]{[#1]}
\def\beginrefs{\begin{list}%
        {[\arabic{equation}]}{\usecounter{equation}
         \setlength{\leftmargin}{2.0truecm}\setlength{\labelsep}{0.4truecm}%
         \setlength{\labelwidth}{1.6truecm}}}
\def\endrefs{\end{list}}
\def\bibentry#1{\item[\hbox{[#1]}]}

%Use this command for a figure; it puts a figure in wherever you want it.
%usage: \fig{NUMBER}{SPACE-IN-INCHES}{CAPTION}
\newcommand{\fig}[3]{
			\vspace{#2}
			\begin{center}
			Figure \thelecnum.#1:~#3
			\end{center}
	}
% Use these for theorems, lemmas, proofs, etc.
\newtheorem{theorem}{Theorem}[lecnum]
\newtheorem{lemma}[theorem]{Lemma}
\newtheorem{proposition}[theorem]{Proposition}
\newtheorem{claim}[theorem]{Claim}
\newtheorem{corollary}[theorem]{Corollary}
\newtheorem{definition}[theorem]{Definition}
\newenvironment{proof}{{\bf Proof:}}{\hfill\rule{2mm}{2mm}}

% **** IF YOU WANT TO DEFINE ADDITIONAL MACROS FOR YOURSELF, PUT THEM HERE:

\newcommand\E{\mathbb{E}}
\newcommand{\bitm}{\begin{itemize}}
\newcommand{\eitm}{\end{itemize}}
\newcommand{\blst}{\begin{lstlisting}}
\newcommand{\elst}{\end{lstlisting}}
\newcommand{\bfig}{\begin{figure}}
\newcommand{\efig}{\end{figure}}


\renewcommand{\thesection}{\arabic{section}}
\renewcommand{\thesubsection}{\thesection.\arabic{subsection}}
\renewcommand{\thesubsubsection}{\thesubsection.\arabic{subsubsection}}

\begin{document}
%FILL IN THE RIGHT INFO.
%\lecture{**LECTURE-NUMBER**}{**DATE**}{**LECTURER**}{**SCRIBE**}
\lecture{3}{Bo Y.-C. Ning}{Bo Y.-C. Ning}{}
%\footnotetext{These notes are partially based on those of Nigel Mansell.}

% **** YOUR NOTES GO HERE:
Due {\bf March 03, 2023} by 11:59pm. 

A few notes:
\begin{enumerate}
\item Submit your homework using the file name "{\bf LastName\_FirstName\_hw3}"

\item Answer all questions with complete sentences. 

\item Your code should be readable; writing a piece of code should be compared to writing a page of a book. Adopt the {\bf one-statement-per-line} rule. Consider splitting a lengthy statement into multiple lines to improve readability. (You will lose one point for each line that does not follow the one-statement-per-line rule)

\item To help understand and maintain code, you should always add comments to explain your code. (homework with no comments will receive 0 points). For a very long comment, break it into multiple lines.

\item Submit your final work with one {\bf .pdf} (or {\bf .html}) file to Canvas. I encourage you to use \href{http://www.docs.is.ed.ac.uk/skills/documents/3722/3722-2014.pdf}{\LaTeX} for writing equations. Handwriting is acceptable, you have to scan it and then combine it with the coding part into a single .pdf (or .html) file. Handwriting should be clean and readable.

\item For $\mathsf{Jupyter \ Notebook}$ users, put your answers in new cells after each exercise. You can make as many new cells as you like. Use code cells for code and Markdown cells for text. 

\item This assignment will be graded based on how you implement your code . 
\end{enumerate}


%%%%%%%%%%%%%%%%%%%%%%%%%%%%%%%%%%%%%%%%%%%%%%%
%%%%%%%%%%%%%%%%%%%%%%%%%%%%%%%%%%%%%%%%%%%%%%%
%%%%%%%%%%%%%%%%%%%%%%%%%%%%%%%%%%%%%%%%%%%%%%%
In this homework, you will work with the Google PageRank problem. You will need to compare the computational speed between direct methods to iterative methods. 


{\Large \bf Questions:}

\begin{enumerate}

\item Open the ucd-web folder from Piazza webpage. The folder contains
two files U.txt and A.txt. U.txt lists the 500 URL names. A.txt is the $500 \times 500$
connectivity matrix. Read data into R or python. Once you read in the data, {\bf take the transpose} of the dataset of A.txt to obtain the A matrix. 

Compute summary statistics:
\begin{enumerate}
\item number of pages
\item number of edges (page links)
\item number of dangling nodes
\item max in-degree
\item max out-degree
\item visualize sparsity pattern of A
\end{enumerate}


\item Set the teleportation parameter at $p = 0.85$. Try the following methods to solve for $x$ using
the ucd-web data and report the speed of each method. You may want to remove some strange URLs (it depends on you how to remove them as long as it makes sense) 
\begin{enumerate}
\item Dense linear system solver: LU decomposition 
\item Dense linear system solver: QR factorization
\item A simple iterative linear system solver such as Jacobi or Gauss-Seidel
\item Choose either a dense eigen-solver such as SVD or iterative method such as the power method
\item Comparing the computational speed for all the methods
\item List the top 20 ranked URLs you found for each method and comment on your findings.

\end{enumerate}


\item As of Monday, 13 Feb 2023, there are at least 4.61 billion indexed webpages on internet according
to \url{http://www.worldwidewebsize.com/}. Comment on whether each of these methods
may or may not work for the PageRank problem at this scale.

\end{enumerate}


%%%%%%%%%%%%%%%%%%%%%%%%%%%%%%%%%%%%%%%%
%%%%%%%%%%%%%%%% Bibliography %%%%%%%%%%%%%%%%%
%%%%%%%%%%%%%%%%%%%%%%%%%%%%%%%%%%%%%%%
\bibliographystyle{chicago}
\bibliography{citation.bib}

 \end{document}